We briefly recall the basic features of \abcST ~\cite{itt20}.
%
Basically, \abcST are abstractions blending together \emph{global
  choreographies}~\cite{gt18,gt16,bcgmmt19}.
%
We use a lightweight visual notation instead of a formal
presentation.
\hsl
%
To this purpose, we use diagrams such as the one in \cref{fig:smp},
depicting the communication pattern of the SM protocol.
%
We now survey the core features of \abcST.
%
\begin{figure}[t]
  \centering
  \begin{tikzpicture}[node distance=.4cm and .6cm, scale=.75, transform shape]
    \mkabcint{}{propose}[m][][{\abctuple[{\abcval[p],\abcget[M][id]}]}][{\abctuple[{\abcval[p],\abcvar}]}][w][{\abcattr[id] = \abcval[hd](\abcget[m][prefs])}]
	 ;
	 \node[below=of propose](branch){};
	 \mkabcint{below left= of branch}{old}[W][][{\abctuple[{\abcval[no]}]}][{\abctuple[{\abcval[no]}]}][][{\abcattr[id]}=\abcvar]
	 \mkabcint{below right= of branch}{new}[W][][{\abctuple[{\abcval[no]}]}][{\abctuple[{\abcval[no]}]}][][{\abcattr[id]=\abcget[w][partner]}]
	 ;
	 \mkbranch{old,new}[branch][][-.8];
	 \mkmerge{old,new}[merge];
	 \mknseq{propose,branch};
	 \mkloop[][.1]{propose}{merge};
  \end{tikzpicture}
  \caption{\label{fig:smp}A behavioural abstraction of the SM protocol}
\end{figure}

The diagram in \cref{fig:smp} represents a distributed workflow of
\emph{interactions}, which are the atomic elements of \abcST; the
general form of an interaction for CAS is inspired by \abc and takes
the form
%
\begin{align}\label{eq:int}
  \abcint
\end{align}
The intuitive meaning of the interaction in~\eqref{eq:int} is as
follows:
\begin{quote}
  \emph{any} agent, say \p, \emph{satisfying} $\abccond$ generates an
  expression $\abcexp$ for \emph{any} agents satisfying $\abccond'$,
  dubbed \q, provided that expression $\abcexp'$ \emph{matches}
  $\abcexp$.
\end{quote}
%
We anticipate that \p\ and \q\ here are used just for convenience and
may be omitted for instance writing $\abcint[]$ or
$\abcint[][@][@][@][]$.
%
Also, we abbreviate $\abcptp$ with $\abcptp[@][]$ when $\abccond$ is a
tautology.
%
This allows us to be more succinct.
%
For instance, the top-most interaction in \cref{fig:smp} specifies
that
\begin{itemize}
\item all \quo{male} agents propose to the \quo{woman} agent who is
  top in their preference list by generating a tuple made of the
  constant value $\abcval[p]$ and their identifier $\abcget[m][id]$
\item any woman agent whose identifier equals
  $\abcval[hd](\abcget[m][prefs])$ pattern matches the tuple sent by
  $\p[m]$ and records in the variable $\abcvar$ the identity of the
  proposer; in the rest of the diagram \p[w] denotes any such woman.
\end{itemize}

Interactions such as \eqref{eq:int} introduce already some
abstractions:
\begin{itemize}
\item Similarly to global choreographies, \eqref{eq:int}
  abstracts away from asynchrony; at the level of this behavioural
  abstraction interactions are supposed to be atomic.
\item As in klaimographies, \eqref{eq:int} abstracts away
  from the number of agents executing the protocol.
\item As behavioural types,  \eqref{eq:int} abstracts away
  from the local behaviour of the agents; for instance,
  it does not tell if/how the local state of agents are
  changed (if at all) by the interaction.
\end{itemize}

Interactions are the basic elements of an algebra of
protocols~\cite{gt18} that allows us to iterate them or compose them
in non-deterministic choices or in parallel.
%
In \cref{fig:smp}, the $\lgateG$-gates mark the entry and exit points
of the loop of the SM protocol.
%
Likewise, the $\orgateG$-gates mark the branching and merging points
of a non-deterministic choice.
%
The \quo{body} of the loop in \cref{fig:smp} yields the choice
specified by the SM protocol.
%
We will also use $\andgateG$-gates to mark forking and joining points
of parallel protocols.

These abstractions are \quo{compatible} with the pseudocode in
\cref{sec:intro}.
%
In fact, one could assign\footnote{
  %
  The assignment should be done with a suitable typing discipline
  which is out of the scope of this paper.
  %
} \mintinline{python}{B} the communication
pattern of \p[m] in \cref{fig:smp} and to \mintinline{python}{C} the
communication pattern of \p[w].
%
Besides this, other \quo{implementations} of the SM procotol could
be given those types.
%
For instance, \p[m] can also be the type of an agent, say
\mintinline{python}{R}, that puts back the identifier of a charger in
its list of preferences once the charger denies the service (so to try
again later).

%%% Local Variables:
%%% mode: latex
%%% TeX-master: "main"
%%% End:
