Increasingly \emph{Collective adaptive systems} (CAS) crop up in many
application domains, spanning critical systems, systems assisting
humans during their working or daily live activities, smart cities.
%
A paradigmatic example is the use artificial autonomous agents in
rescue contexts that may put operators lives at \eMcomm[citare
qualcosa?]{stake.}
%
These agents execute in a cyber-physical context and are supposed to
exhibit an \emph{adaptive} behaviour.
%
This adaptation should be driven by the changes occurring in their
operational environments, besides by the changes in the local state of
computation of components, \quo{collectively taken}.
%
Also, the \emph{global} behaviour of CAS should \emph{emerge} from the
\emph{local} behaviour of its components.
%
Let us explain this considering the coordination of a number of robots
patrolling some premises to make sure that aid is promptly given to
human operators in case of accidents.

A plausible local behaviour of each robot can be:
\begin{enumerate*}[label=(\arabic*)]
\item\label{it:accident} to identify accidents,
\item\label{it:assess} to assess the level of gravity of the situation
  (so to choose an appropriate course of action),
\item\label{it:alert} to alert the rescue centre and nearby robots (so
  to e.g, divert traffic to let rescue vehicles reach the location of
  the accident more fastly), and
\item\label{it:state} to ascertain how to respond to alerts from other
  robots (e.g., if already involved in one accident or on a low
  battery, a robot my simply forward the alter to other nearby
  robots).
\end{enumerate*}
%
Note that robots' behaviour depend on the physical environment (tasks
\cref{it:accident,it:assess,it:alert}) as well as its current state in
\cref{it:state}.

A possible expected global behaviour is that robots try maximise the
patrolled area while trying to avoid remaining isolated and to minimise
the battery consumption.
%
It is worth remaking that the global behaviour is not typically
\emph{explicitly} programmed; it rather should be the resultant
of the combination of the behaviour of the single components.
%
For instance, when designing the algorithm for the roaming one
could assume that a robot does not move towards an area where
there already a certain number of robots.

This paper applies behavioural specifications to the quantitative
analysis of CAS.
%
Using a simple, yet representative scenario, we show how to use
behavioural specifications and use them to study non-functional
properties of CAS (emergent) behaviour.
%
More precisely, we use Petri nets and behavioural types to model a
robot scenario recently proposed in~\cite{itt20}.
%
These two models are rather different and their application allows
us to make a comparison between the two approaches.

\paragraph{Outline.} \eMcomm[todo]{}

%%% Local Variables:
%%% mode: latex
%%% TeX-master: "main"
%%% End:
